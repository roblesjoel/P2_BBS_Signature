\documentclass{article}

% Language setting
% Replace `english' with e.g. `spanish' to change the document language
\usepackage[english]{babel}

% Set page size and margins
% Replace `letterpaper' with `a4paper' for UK/EU standard size
\usepackage[a4paper,top=2cm,bottom=2cm,left=3cm,right=3cm,marginparwidth=1.75cm]{geometry}

% Useful packages
\usepackage{amsmath}
\usepackage{graphicx}
\usepackage[colorlinks=true, allcolors=blue]{hyperref}

\title{Top Level of BBS}
\author{Joel Gabriel Robles Gasser \& Miguel Angel Schweizer}

\begin{document}
\maketitle

\begin{abstract}
\end{abstract}

\section{Introduction}


\section{Building Blocks}

\subsection{Self-Sovereign Identity}

Self-sovereign identity is a model managing digital identities. In this model the core concept is that individuals or coorperations have the complete ownership and control of their accounts and personal data.

\subsection{Verifiable Credentials}

Verifiable credentials, also known as VCs, are a W3C standard for digital, cryptographically verifiable credentials. The VCs are stored on digital devices and use cryptography to verify the data and the authorship of the credentials.

\subsection{Trust Triangle}

The trust triangle is a very important model for the BBS signature scheme, which provides the characteristics to the scheme and makes it work. 
We have three-way relationship between parties which are issuer, holder and verifier. 
The signer signs the messages from the holder and sends them back.
The holder handles proof generation of the signature and the messages. He sends the messages he wants to have signed to the signer. He also sends the generated proof and the disclosed messages to the verifier.
The verifier verifies the proof he receives from the holder.
%The holder trusts the signer because he sends him his messages to be signed. The verifier trust the signer because he signed the messages of the holder. 
%That is how the triangle of trust works. 

\subsection{Proof of knowledge \& Zero Knowledge Proof}

A lower version of zero-knowledge proof is proof-of-knowledge. Proof-of-knowledge has the two characteristics of completeness and soundness. Completeness means that if the proof is true, thus the verifier can be convinced by the holder that they possess knowledge about the correct proof. Soundness is the opposite of completeness, if the proof is false, then no holder can convince a verifier that they possess knowledge about the proof. Zero-knowledge-proof adds a characteristic known as zero-knowledge. This says that if a proof is true, then the verifier learns nothing more from the holder other than the proof is true.

\section{Properties of BBS}

\subsection{Selective Disclosure}

With the BBS signature scheme we can sign multiple messages at once. This allows the holder to generate a proof of the messages and the signature. The holder itself can choose which message he wants to disclose, while revealing no-information about the undisclosed messages. The proof is a guarantee of integrity and authenticity of the disclosed message.

\subsection{Unlinkable Proofs}

The holder generates a proof, which is known as a zero-knowledge proof. In our case this means that a verifier cannot determine which signature was used to generate the proof. This removes a common source of correlation. Each proof generated is indistinguishable from random even for two proofs generated from the same signature. 

\subsection{Proof of Possession}

The proof of possession means that the proofs generated by the holder, prove to a verifier that the holder was in possession of a signature without revealing the signature itself. The BBS signature scheme supports the binding of headers to the generated proof which hold additional information. 

\subsection{Link Secrets}

Link secrets enable more trusted interactions with verifiable credentials. With a cryptographic commitment algorithm the holder creates a commitment, which allows the holder to prove his/her knowledge of the secret without revealing it. The holder can send the commitment to the signer, which signs it together with the credential attributes, while the secret is never shared. The link secret is a large random number and called master secret. 
It is called link secret, because the master secret is inserted and signed in different credentials. By adding it in every credential, it links the holder to the credentials and link the credentials to each other. The master secret is randomized by a blinding algorithm at every issuance of a credential. The master secret is kept secret at every time and is never revealed by the holder. Only the commitments and proofs of knowledge of the committed secret are shared.

%\cite{latex2e}
\bibliographystyle{alpha}
\bibliography{references}

\end{document}